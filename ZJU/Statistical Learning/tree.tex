\section{Tree and Random Forest}

Nonparametric methods

\subsection{Splitting rules} % (fold)
\label{sub:Splitting rules}

$$
Gini(\mathcal{D}) = 1 - \sum_{k=1}^{K} p_k^2
$$
$$
Score = Gini(\mathcal{T}) - (\dfrac{N_{\mathcal{T}_L} }{N_{\mathcal{T}}} Gini(\mathcal{T}_L) + \dfrac{N_{\mathcal{T}_R}}{N_{\mathcal{T}}} Gini(\mathcal{T}_R))
$$
Go through all variables j and all cutting points c to find the split with the best score.
$$
(ID3,C4.5)Entropy = \sum\limits_{k=1}^{N} -p_k log(p_k)
$$
$$
Error = 1 - max(p_k)
$$

Gini and Shannon are more sensitive to changes. Maximum of $2^{M-1}-1  $ of possible splits.

For regression trees:
$$
score = Var(\mathcal{T}) - (\dfrac{N_{\mathcal{T}_L}}{N_{\mathcal{T}}} Var(\mathcal{T}_L) + \dfrac{N_{\mathcal{T}_R}}{N_{\mathcal{T}}} Var(\mathcal{T}_R)) \qquad Var(\mathcal{T}) = \dfrac{1}{N_{\mathcal{T}}} \sum\limits_{i=1}^{N_{\mathcal{T}}} (y_i - \bar y)^2
$$

For any sub-tree of $T_{max}$ denote as $T \preceq T_{max} $ calculate
$$
C_\alpha (T)  = \sum\limits_{\text{all termial nodes t in T}}^{} N_t \cdot Impurity(t) + \alpha |T|
$$
where $|T|$ is the number of terminal nodes in $T$, $\alpha$ is a penalty parameter.

$$
\alpha \le \dfrac{C(t) - C(T_t)}{|T_t|-1}
$$
逐渐删掉最小的 $\alpha$

Tree methods can handle missing value by either putting them as a separate category or using surrogate variables.


Pros and cons of tree methods:
\begin{itemize}
    \item Pros:
    \begin{itemize}
        \item Easy to interpret and visualize.
        \item Can handle both numerical and categorical data.
        \item Non-parametric, no assumptions about the distribution of the data.
        \item Can capture non-linear relationships.
    \end{itemize}
    \item Cons:
    \begin{itemize}
        \item Prone to overfitting, especially with deep trees.
        \item Sensitive to small changes in the data.
        \item Can be biased towards features with more levels (in categorical variables).
        \item not smooth
    \end{itemize}
\end{itemize}


% subsection Splitting rules (end)

\subsection{Random Forest} % (fold)
\label{sub:Random Forest}

不仅仅是多个树的叠加,feature的选取也是随机的。

Bootstrap

mtry: At each split, randomly select mtry variables from the entire set of features.

nmin: split until the number of samples in a node is less than nmin.

ntree: number of trees to grow.

\paragraph{Variable Importance} % (fold)

For each tree $m$, use the oob as the testing set to obtain, obtain the prediction error: $Err_0^m$

For each variable $j$, permute the values of $j$ in the oob set, and obtain the prediction error: $Err_j^m$.

$$
VI_{mj} = \dfrac{Err^m_j}{Err^m_0} - 1
$$
$$
VI_j = \dfrac{1}{M} \sum\limits_{m=1}^{M} VI_{mj}
$$
\begin{figure}[H]
\centering
\includegraphics[width=0.8\textwidth]{figure/VI.png}
\caption{Variable Importance}
\label{fig:Variable Importance}
\end{figure}
% subsection Random Forest (end)


\subsection{Adaptive Kernel} % (fold)
\label{sub:Adaptive Kernel}
$$
\hat f(x_0) = \dfrac{\sum\limits_{i=1}^{} y_i K_{\lambda} (x_0,x_i)}{\sum\limits_{i=1}^{} K_{\lambda} (x_0,x_i)}
$$

In random forest, the kernel is adaptive to the local structure of the data. The bandwidth $\lambda$ can be adjusted based on the density of points in the neighborhood of $x_0$.

$$
\mathcal{A}(x_i,x_0)  = \sum\limits_{k \in \mathcal{K}}^{} \mathbf{1}{x_i \in A_k} \cdot \mathbf{1}{x_0 \in A_k} = \begin{cases}
1 & \text{if } x_i \text{ and } x_0 \text{ are in the same terminal node} \\
0 & \text{otherwise}
\end{cases}
$$

\begin{center}
    Tree estimator $\hat f(x_0) = \dfrac{\sum\limits_{i=1}^{} y_i \mathcal{A}(x_i,x_0)}{\sum\limits_{i=1}^{} \mathcal{A}(x_i,x_0)}$\\
    Random Forest estimator $\hat f(x_0) = \dfrac{\sum\limits_{m=1}^{ntree} \sum\limits_{i=1}^{} y_i^m \mathcal{A}(x_i^m,x_0)}{\sum\limits_{m=1}^{ntree} \sum\limits_{i=1}^{} \mathcal{A}(x_i^m,x_0)}$
\end{center}


\textbf{Random Forests is a kernel method with adaptive bandwidth:}

• Each tree in a forest defines a kernel: uniform within each terminal node

• Ensemble: sum of kernels is still a kernel

• A tree is “more likely” to split on important variables, making their “bandwidth” smaller

• If the splitting variables are selected wisely, the bandwidth is adaptive to the signal strength

• The cutting point is more likely to happen on the zero curvature of the underlying target function
% subsection Adaptive Kernel (end)


U statistics:
