\section{Support Vector Machines (SVM)}
Support Vector Machines (SVM) are a powerful class of supervised learning algorithms used for classification and regression tasks. 
They work by finding the optimal hyperplane that separates different classes in the feature space.

$y_i \in \{-1,1\}$

\subsection{separable} % (fold)
\label{sub:separable}
Maximize the separation margin:

$$
\begin{aligned}
\max_{\boldsymbol{\beta}, \beta_0, \|\beta\| = 1} \quad & M \\
\text{s.t.} \quad & y_i (x_i^\top \boldsymbol{\beta}+\beta_0) \geq M, \quad i = 1, \ldots, n
\end{aligned}
$$
\begin{figure}[h]
    \centering
    \includegraphics[width=0.5\textwidth]{figure/cut_plane.png}
    \caption{SVM with separable data}
    \label{fig:svm_separable}
\end{figure}

$$
\min_{\boldsymbol{\beta},\beta_0} \quad \frac{1}{2} \|\boldsymbol{\beta}\|^2 
\text{ subject to } y_i (x_i^\top \boldsymbol{\beta}+\beta_0) \geq 1, \quad i = 1, \ldots, n
$$

Consider the Lagrangian:
$$
\mathcal{L}(\boldsymbol{\beta}, \beta_0, \boldsymbol{\alpha}) = \frac{1}{2} \|\boldsymbol{\beta}\|^2 - \sum_{i=1}^{n} \alpha_i (y_i (x_i^\top \boldsymbol{\beta}+\beta_0) - 1)
$$

Dual problem:
$$
\max_{\boldsymbol{\alpha}} \quad \sum_{i=1}^{n} \alpha_i - \frac{1}{2} \sum_{i=1}^{n} \sum_{j=1}^{n} \alpha_i \alpha_j y_i y_j (x_i^\top x_j)
\text{ subject to } \sum_{i=1}^{n} \alpha_i y_i = 0, \quad \alpha_i \geq 0, \quad i = 1, \ldots, n
$$

The two problem are the same if 1. both g and h are convex, 2. the constraint h are feasible.

Advantages of the duality:
\begin{itemize}
    \item SMO algorithms
    \item Change the problem from p dimension into n dimension.
    \item Kernel trick: we can use the kernel function to transform the data into a higher-dimensional space without explicitly computing the coordinates in that space.
\end{itemize}

\begin{figure}[htbp]
\centering
\includegraphics[width=0.8\textwidth]{figure/svm_procedure.png}
\end{figure}

线性可分的时候,logistic的likelihood是正无穷,不好。

当满足高斯分布的时候,lda是最好的,但是这个条件苛刻。

% subsection separable (end)

\subsection{Nonseparable SVM} % (fold)
\label{sub:Nonseparable SVM}
引入松弛变量
$$
\begin{aligned}
\min_{\boldsymbol{\beta}, \beta_0, \boldsymbol{\xi}} \quad & \frac{1}{2} \|\boldsymbol{\beta}\|^2 + C \sum_{i=1}^{n} \xi_i \\
\text{s.t.} \quad & y_i (x_i^\top \boldsymbol{\beta}+\beta_0) \geq 1 - \xi_i, \quad i = 1, \ldots, n \\
& \xi_i \geq 0, \quad i = 1, \ldots, n
\end{aligned}
$$
C 越大,尽量分对。 C越小,尽量扩大间隔。

The Lagrangian is:
$$
\mathcal{L}(\boldsymbol{\beta}, \beta_0, \boldsymbol{\xi}, \boldsymbol{\alpha}, \boldsymbol{\nu}) = \frac{1}{2} \|\boldsymbol{\beta}\|^2 + C \sum_{i=1}^{n} \xi_i - \sum_{i=1}^{n} \alpha_i (y_i (x_i^\top \boldsymbol{\beta}+\beta_0) - 1 + \xi_i) - \sum_{i=1}^{n} \gamma_i \xi_i
$$
$\alpha_i,\gamma_i > 0$

Gradient:
$$
\begin{aligned}
\nabla_{\boldsymbol{\beta}} \mathcal{L} &= \boldsymbol{\beta} - \sum_{i=1}^{n} \alpha_i y_i x_i = 0 \\
\nabla_{\beta_0} \mathcal{L} &= -\sum_{i=1}^{n} \alpha_i y_i = 0 \\
\nabla_{\boldsymbol{\xi}} \mathcal{L} &= C - \boldsymbol{\alpha_i} - \boldsymbol{\gamma_i} = 0
\end{aligned}
$$

Dual problem:
$$
\max_{\boldsymbol{\alpha}, \boldsymbol{\gamma}} \quad \sum_{i=1}^{n} \alpha_i - \frac{1}{2} \sum_{i=1}^{n} \sum_{j=1}^{n} \alpha_i \alpha_j y_i y_j  \langle x_i, x_j\rangle
\text{ subject to } \sum_{i=1}^{n} \alpha_i y_i = 0, \quad 0 \leq \alpha_i \leq C, \quad i = 1, \ldots, n
$$

Useless data points are those with $\alpha_i = 0$.

Support vectors are those with $\alpha_i > 0$.

1. $0<\alpha_i < C$ and $\xi_i = 0$

2. $\alpha_i = C$ and $\xi_i= 1- y_i(x_i^\top \boldsymbol{\beta} +\beta_0) > 0$



\subsection{Kernel Trick} % (fold)
\label{sub:Kernel Trick}

我们只关心映射到另一个空间之后的样本之间的内积,而不是映射到另一个空间之后的样本本身。

The kernel trick allows us to compute the inner product in a high-dimensional space without explicitly mapping the data points to that space. This is particularly useful when dealing with non-linear decision boundaries.

The kernel function $K(x_i, x_j)$ is defined as:
$$
K(x_i, x_j) = \langle \phi(x_i), \phi(x_j) \rangle
$$
where $\phi(x)$ is a mapping function that transforms the input data into a higher-dimensional space.

Common kernel functions include:

\begin{itemize}
    \item Linear kernel: $K(x_i, x_j) = x_i^\top x_j$
    \item Polynomial kernel: $K(x_i, x_j) = (x_i^\top x_j + c)^d$, where $c$ is a constant and $d$ is the degree of the polynomial.
    \item Radial basis function (RBF) kernel: $K(x_i, x_j) = \exp(-\gamma \|x_i - x_j\|^2)$, where $\gamma$ is a parameter that controls the width of the Gaussian function.
\end{itemize}

% subsection Kernel Trick (end)

\subsection{Convexity of SVM} % (fold)
\label{sub:Convexity of SVM}

$$
\sum\limits_{i,j = 1}^{n} y_i y_j \alpha_i \alpha_j K(x_i, x_j) = \alpha^\top diag(y) K diag(y) \alpha
$$

Convexity will be guaranteed if the Kernel matrix K is positive semi definite.

\textbf{Mercer's theorem} states that a kernel function is positive semi-definite if it can be expressed as an inner product in some feature space.

% subsection Convexity of SVM (end)


\subsection{Penalized version of svm} % (fold)
\label{sub:Penalized version of svm}

Recall that the prime objective function for SVM is:
$$
\min_{\boldsymbol{\beta}, \beta_0} \quad \frac{1}{2} \|\boldsymbol{\beta}\|^2 + C \sum\limits_{i=1}^{n} \xi_i
$$
$$
\text{s.t. } y_i (x_i^\top \boldsymbol{\beta}+\beta_0) \geq 1 - \xi_i, \quad i = 1, \ldots, n \qquad
\xi_i \ge 0, \quad i = 1, \ldots, n
$$

$$
\min_{\boldsymbol{\beta}, \beta_0} \quad \frac{1}{2} \|\boldsymbol{\beta}\|^2 + C \sum\limits_{i=1}^{n} \max(0, 1 - y_i (x_i^\top \boldsymbol{\beta}+\beta_0))
$$

Other forms of loss function:
\begin{enumerate}
    \item Logistic loss: $log(1+e^{-yf(x)} )$
    \item Modified Huber Loss: 
$L = \begin{cases}
    max(0,1-yf(x))^2, & \text{for } yf(x) \ge -1 \\ 
    -4yf(x), & \text{o.w. }  \\
\end{cases}$
    

\end{enumerate}
Some of which is differentiable, like logistic,modified huber, squared.
Hinge loss is not differentiable at 0, but subgradient is available.
0,1 loss is not differentiable.





% subsection Penalized version of svm (end)









