
\section{Classification}
$$
y_i \in \{-1,1\} or \{0,1\}
$$
$$
0-1 Loss L(y,f(x)) := \begin{cases}
    0 & \text{if } y = f(x) \\
    1 & \text{if } o.w.
\end{cases}
$$

Soft classifier: estimate the conditional probability. Logistic

Hard classifier: estimate the class label directly. SVM

$$
p(Y = y_i | X = x_i) = \eta(x_i)^{y_i} (1 - \eta(x_i))^{1-y_i} 
$$
$$
where \log(\dfrac{\eta}{1-\eta}) = \beta^T x_i
$$
log odds: $log \dfrac{p}{1-p} $, $\eta(x) = \dfrac{exp(x^\top \beta)}{1+exp(x^\top \beta)}$

Use mle to estimate the parameters $\beta$:
$$
l = \sum\limits_{i=1}^{n} \log p(y_i|x_i,\beta)
$$
Use newton's method to find the maximum likelihood estimate of $\beta$.
$$
\beta^{new} = \beta^{old} - [\dfrac{\partial^2 l}{\partial \beta \partial \beta ^\top}]^{-1} \dfrac{\partial l}{\partial \beta} |_{\beta^{old}} 
$$

Each unit increases in $X_j$ increases the log-odds of Y by $\beta_j$

\begin{figure}[h]
    \centering
    \includegraphics[width=0.5\textwidth]{figure/type_error.png}
    \caption{Logistic Regression}
    \label{fig:logistic_regression}
\end{figure}

第一类错误(假阳性):健康人误诊为患者,相当于无罪的人被判有罪

第二类错误(假阴性):患者漏诊为健康人,相当于有罪的人被判无罪

$$
Overall \ Error = P(\text{假阳性}) + P(\text{假阴性}) = P(\text{假阳性}) + (1 - P(\text{真阳性}))
$$
$$
Sensitivity / Recall = P(\text{真阳性}) = \dfrac{TP}{TP + FN} = \dfrac{TP}{P(\text{患者})} = 1 - Type \ 2 \ Error
$$
$$
Specificity = P(\text{真阴性}) = \dfrac{TN}{TN + FP} = \dfrac{TN}{P(\text{健康人})} = 1- Type\ 1 \ Error
$$

\begin{figure}[h]
    \centering
    \includegraphics[width=0.8\textwidth]{figure/ROC.png}
    \caption{Chaneg different threshold to get different sensitivity and specificity}
    \label{fig:roc_curve}
\end{figure}

\subsection{LDA and QDA} % (fold)
\label{sub:LDA and QDA}

Bayes rule:
$$
P(Y = 1 | X = x) = \dfrac{P(X = x | Y = 1) P(Y = 1)}{P(X = x)}
$$

Treat $\pi = P(Y=1)$ as prior probabilities,and $f_1 = P(X=x|Y=1)$ and $f_0 = P(X=x|Y=0)$.

$$
f_B(x) = arg\min_{f} R(f) = \begin{cases}
    1 & \text{if } f_1(x) \pi > f_0(x) (1-\pi) \\
    0 & \text{if } f_1(x) \pi < f_0(x) (1-\pi)
\end{cases}
$$

Multi-class
$$
f_B(x) = arg\max_{k}P(Y = k|X = x)  = arg \max_k \pi_k f_k(x)
$$


Masking problem: linear polynomial may not perform well.

$$
P(Y=k|X=x) = \dfrac{P(X=x|Y=k)P(Y=k)}{\sum\limits_{j=1}^{K} P(X=x|Y=j)P(Y=j)} = \dfrac{f_k(x)\pi_k}{\sum\limits_{j=1}^{K} f_j(x)\pi_j}
$$

P26 

$$
\hat f(x) = arg\max_k log(\pi_k f_k(x)) = arg \max -\dfrac{1}{2}(x-\mu_k)^T \Sigma^{-1} (x-\mu_k) + log(\pi_k)
$$

Note that we only care about terms related to $k$.

The discriminant function is:
$$
\delta_k(x) = x^\top \Sigma^{-1} \mu_k -\dfrac{1}{2}\mu_k^T \Sigma^{-1} \mu_k + log(\pi_k)
$$ then we could calculate the decision boundary by setting $\delta_k(x) = \delta_j(x)$ for $k \neq j$.

如何计算先验的概率 p31

QDA simply abandons the assumption of equal covariance matrices for all classes, allowing each class to have its own covariance matrix $\Sigma_k$.
% subsection LDA and QDA (end)

\begin{figure}[h]
    \centering
    \includegraphics[width=0.8\textwidth]{figure/qda.png}
    \caption{QDA decision boundary}
    \label{fig:lda_qda}
\end{figure}

Comparison of LDA and QDA:
\begin{itemize}
    \item LDA assumes equal covariance matrices for all classes, while QDA allows each class to have its own covariance matrix.
    \item More paremeters in QDA
    \item Both are easy to implement.
    \item We could select quadratic terms and perform lda.
    \item p is large, the inverse of $\Sigma$ might not exist.
\end{itemize}

\subsection{Alternative Methods} 

Fisher criterion: maximize the ratio of between-class variance to within-class variance.

$$
B = \sum\limits_{k=1}^{K} \pi_k (\mu_k - \bar{\mu})(\mu_k - \bar{\mu})^\top
$$
where $\bar{\mu} = \sum\limits_{k=1}^{K} \pi_k \mu_k$ is the overall mean vector.

Denote the within class covariance matrix as $W$, which is common.

$$
\max_a \dfrac{a^\top B a}{a\top W a}
$$
The problem is equivalent to finding the eigenvector corresponding to the largest eigenvalue of the matrix $W^{-1}B$.

Regularization: sparse lda, 加一个 L1 penalty of vector a.

Regularized da (RDA): 在每个类的协方差矩阵上加一个 common的矩阵。

Naive Bayes: 把每个特征都作为独立的, $f_k(x) \approx \Pi_{j=1} ^p f_{kj} (x_j)$

\vspace{20pt}
Comparison between LDA and logistic regression:
\begin{itemize}
    \item LDA assumes normality and equal covariance matrices, while logistic regression does not.
    \item LDA is a generative model, while logistic regression is a discriminative model.
    \item LDA can be more robust to outliers due to its assumptions.
    \item Logistic regression is easier to interpret in terms of odds ratios.
\end{itemize}