% TeX root = ../Main.tex

% First argument to \section is the title that will go in the table of contents. Second argument is the title that will be printed on the page.
\section[Lecture 14 (Oct 28)-- {\it Posterior Distribution}]{Lecture 14}

\subsection{Bayesian}


For rejection sampling, 
the $\theta_{i} ^{F} := $ the $i$-th item from final list of $B$ size.
\\It follows some distribution $q$

\[
q(\theta) = \dfrac{f(\theta) f(X_1,\dots,X_n | \theta)}{f(X_1,\dots,X_n)}
\]
\[
q(\theta) \propto f(\theta) \dfrac{L_n(\theta)}{L_n(\hat \theta _n)} \propto f(\theta) f(X_1,\dots,X_n|\theta) \propto f(\theta|X_1,\dots,X_n)
\]

\subsection{Importance Sampling} % (fold)

\[
\begin{aligned}
E_h[q(\theta)]
&= \int q(\theta) h(\theta) \, d\theta \\[6pt]
&= \int q(\theta) \frac{h(\theta)}{g(\theta)} g(\theta) \, d\theta \\[6pt]
&\approx \frac{1}{B} \sum_{i=1}^B q(\theta_i) \frac{h(\theta_i)}{g(\theta_i)}.
\end{aligned}
\]

We choose $h(\theta)$ to be the posterior distribution $f(\theta|X_1,\dots,X_n)$, and $g(\theta)$ to be the prior distribution $f(\theta)$.

Then we have
\[
E_{f(\theta|X_1,\dots,X_n)}[q(\theta)]
\approx \frac{1}{B} \sum_{i=1}^B q(\theta_i) \frac{f(\theta_i|X_1,\dots,X_n)}{f(\theta_i)}
\]

How to get $f(\theta_i|X_1,\dots,X_n)$?

Denote $L_n(\theta) = f(X_1,\dots,X_n|\theta)$, then

Denote the fraction to be :
\[
w_i := \dfrac{f(\theta_i | X_1,\dots,X_n)}{B \cdot f(\theta_i)} 
\]

The numerator \[
f(\theta_i | X_1,\dots,X_n) = \dfrac{f(X_1,\dots,X_n|\theta_i) f(\theta_i)}{f(X_1,\dots,X_n)} = \dfrac{L_n(\theta_i) f(\theta_i)}{f(X_1,\dots,X_n)}
\]

\[
f(X_1,\dots,X_n) = \int f(X_1,\dots,X_n,\theta) d \theta = \int f(X_1,\dots,X_n | \theta) \dfrac{f(\theta)}{g(\theta)} g(\theta) d \theta \approx \dfrac{1}{B} \sum_{j=1}^{B} L_n(\theta_j) \dfrac{f(\theta_j)}{g(\theta_j)}
\]

We can choose the second MC to use the same as the first MC, i.e., $g(\theta) = f(\theta)$

Thus, the weight is :
\[
=  \dfrac{L_n(\theta_i)}{B \cdot \int_{}^{} f(X_1,\dots,X_n | \theta) f(\theta) d\theta} \approx \frac{L_n(\theta_i)}{ \sum\limits_{j=1}^{B} L_n(\theta_j) } 
\]

Finally, we have
\[
\E[q(\theta) | x^n] \approx \sum_{i=1}^B w_i q(\theta_i)
\]
% paragraph Importance Sampling (end)

\subsection{Rejection Sampling} % (fold)
\label{sub:Rejection Sampling}

Goal: get s sample for $r(x)$

Proposal distribution: $g(x)$, which is easy to sample from.

\emph{Rejection procedure:} find  $M \quad \text{s.t.} \quad \forall x, \dfrac{r(x)}{M\cdot g(x)} < 1 $

\begin{enumerate}
    \item Sample $X_{cand}  \sim g(x)$
    \item Sample $U \sim \text{Uniform}(0,1)$
    \item If $U \le \dfrac{r(X_{cand} )}{M \cdot g(X_{cand} )}$, accept $X_{cand} $; else reject $X_{cand} $ and go to step 1.
\end{enumerate}

Following the above procedure, the sample probability density function is:
\[f_{X_{acc}} (x) \propto g(x) \cdot \dfrac{r(x)}{M \cdot g(x)} \propto r(x)\]


\begin{eb}
For posterior distribution sampling, we can set
\[
r(\theta) = f(\theta \mid x), \quad g(\theta) = f(\theta)
\]

We can multiply by some function that does not depend on $\theta$ to make the sampling easier:
\[
\theta_1, \dots, \theta_B \sim r(\theta) \propto r(\theta) f(x) = f(\theta,x) \propto f(x \mid \theta) g(\theta)
\]

We need to find $M$ such that
\[\dfrac{f(x \mid \theta) g(\theta)}{M \cdot g(\theta)} = \dfrac{f(x \mid \theta)}{M} < 1 \quad \forall \theta\]
i.e.
\[M = \sup_\theta f(x \mid \theta) = f(x \mid \hat \theta_{MLE} )\]
\end{eb}

% subsection Rejection Sampling (end)