\section[Lecture 3 (Spet 4) -- {\it Bootstrap}]{Lecture 3}

\begin{tb}
    The Dvoretzky-Kiefer-Wolfowitz Inequality states that for i.i.d. random variables $X_1, \dots, X_n$ with empirical distribution $\hat{F}_n$ and true distribution $F$, the following holds:
$$
P(\underset{x}{\sup} |F(x) - \hat{F}_n(x)| > \epsilon ) \le 2 e^{-2n\epsilon^2} 
$$
\end{tb}

Let the RHS be $1-\alpha \rightarrow \epsilon = \sqrt{\frac{1}{2n} \log \frac{2}{\alpha}}$

Then we have $$
P(\hat{F}_n(x) - \epsilon \le F(x) \le \hat{F}_n(x) + \epsilon, \forall x) \ge 1-\alpha
$$

Let $L(x) = \max\{ \hat{F}_n(x) - \epsilon, 0\}$ and $U(x) = \min\{\hat{F}_n(x) + \epsilon, 1\}$

Then we have $P(L(x) \le F(x) \le U(x), \forall x) \ge 1-\alpha$


Often we have $T(\hat{F}_n) \approx N(T(F),\hat{se}^2)$, which allows us to form an approximate $1-\alpha$ confidence interval.
We need to find an asymptotic distribution of $T(\hat{F}_n)$.

$\theta(F) = T(F)$ quantity of interest (often a single value instead of function like F)

We will have $$
P(|\dfrac{T(f)- T(\hat{F}_n)}{\hat{se}}| \le z_{\alpha/2}) \approx 1-\alpha
$$


And we focus on this interval: 
$$
T(\hat{F}_n) \pm z_{\alpha/2} \hat{se}
$$


\subsection{Bootstrap} % (fold)
\label{sub:Bootstrap}

\subparagraph{Monte Carlo} % (fold)
\label{subp:Monte Carlo}

$$
E(h(Y)) = \int h(y) \, d F_Y(y) \approx \frac{1}{n} \sum_{i=1}^n h(Y_i) \text{ where } Y_i \overset{\text{i.i.d.}}{\sim} F_Y 
$$

Note that if $E[h(Y)] < \infty$, then
$$
RHS \overset{a.s.}{\to} E[h(Y)] \text{ as } n \to \infty
$$

\begin{eb}
    Approx $\int_{-\infty}^{\infty} sin^2(x) e^{-x^2} \, dx$ using Monte Carlo with $n=1000$ samples.
\end{eb}

$$
\sqrt{\pi}\int_{-\infty}^{\infty} sin^2(x) \frac{1}{\sqrt{\pi}} e^{-x^2} \, dx = E[sin^2(X)] \text{ where } X \sim N(0,1/2)
$$

\begin{lstlisting}[language=Python]
import numpy as np
n = 10000
X = np.random.normal(0, np.sqrt(1/2), n)
np.sqrt(np.pi) * np.mean(np.sin(X)**2)
\end{lstlisting}

Even though, the target density is $h$.
More generally, we can use Monte Carlo for:
$$
E_h[q(\theta)] = \int h(\theta) q(\theta) \, d\theta = \int q(\theta) \dfrac{h(\theta)g(\theta)}{g(\theta)}\, d\theta \approx \frac{1}{n} \sum_{i=1}^n \dfrac{h(\theta_i)q(\theta_i)}{g(\theta_i)} \text{ where } \theta_i \overset{\text{i.i.d.}}{\sim} g(\theta)
$$

i.e. we can sample from a different distribution $g$ and use importance weights $\dfrac{q(\theta)}{g(\theta)}$ to adjust.

%% Find 1(x > 3) where x ~ N(0,1)
%% E[1(X>3)] = P(X>3) = \int 1
\begin{lstlisting}[language=Python]
import numpy as np
n = 1000
X = np.random.normal(0, 1, n)
np.mean(X > 3)
# np.float64(0.002)
\end{lstlisting}

Now try to stimualate using importance sampling:
\begin{lstlisting}[language=Python]
import numpy as np
n = 1000
X = np.random.normal(3, 1, n)
np.mean((X > 3) * np.exp(-X**2/2 + (X-3)**2/2))
# np.float64(0.0014236252168949273)
\end{lstlisting}

If we knew F , we could use MC integration to approximate $\var F(T_n)$.
However, we don’t in practice, so we make an initial approximation of F
with the empirical CDF $\hat{F}_n$ and then use MC integration to approximate $V_{\hat{F}_n}[T_n]$.
$$
V_F[T_n] \overset{ECDF}{\approx} V_{\hat{F}_n} \overset{MC}{\approx} \hat{V}_{\hat{F}_n}
$$

