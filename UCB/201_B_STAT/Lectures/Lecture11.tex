% TeX root = ../Main.tex

% First argument to \section is the title that will go in the table of contents. Second argument is the title that will be printed on the page.
\section[Lecture 11 (Oct 7) -- {\it Hypothesis Testing}]{Lecture 11}


\subsection{Hypothesis Testing}

A \textbf{statistical hypothesis} is a statement about a parameter (or a statistical functional in nonparametric models).

A hypothesis test partitions the parameter space $\Theta$ into two disjoint sets $\Theta_0$ and $\Theta_1$, and produces a decision rule for choosing between

\[
H_0 : \theta \in \Theta_0 \quad \text{and} \quad H_1 : \theta \in \Theta_1
\]

$H_0$ is called the \emph{null hypothesis} and $H_1$ is the \emph{alternative hypothesis}.  
The possible choices are:

\begin{itemize}
  \item Reject $H_0$
  \item Fail to reject $H_0$
\end{itemize}

We evaluate a test using its \emph{power function}, defined as
\[
\beta(\theta) = P_\theta(X \in R)
\]




\subsection{Reject Rule} % (fold)
\label{sub:Reject Rule}

The decision of whether to reject $H_0$ is determined by whether the sample $X = (X_1, \ldots, X_n)$ falls into a predefined rejection region $R$.

Usually, the rejection region has the form
\[
R = \{ (x_1, \ldots, x_n) : T(x_1, \ldots, x_n) > c \}
\]
where $T$ is called a \emph{test statistic} and $c$ is the \emph{critical value}.

The idea is to construct $R$ so that the probability of the data falling into it when $H_0$ is true is small.
And the \textbf{test size }would be $\alpha = \sup_{\theta \in \Theta_0} \beta(\theta)$.

\begin{figure}[H]
\centering
\includegraphics[width=0.8\textwidth]{Images/PN.png}
\caption{Positive and Negative Predictive Values vs Prevalence}
\end{figure}

\begin{eb}[Normal distribution]
Suppose $X_1, \ldots, X_n \sim N(\mu, \sigma^2)$, and let $\hat{\mu}_n$ and $\hat{\sigma}^2_n$ be the MLEs.  
If $H_0: \mu = 0$, one test statistic we might consider is $T = |\hat{\mu}_n / \hat{\sigma}_n|$, reasoning that if $H_0$ is true, $T$ will tend to be small.


\bigskip
Let $X_1, \ldots, X_n \sim N(\mu, \sigma^2)$, with $\sigma^2$ known.  

Test $H_0 : \mu = 0$ vs $H_1 : \mu \ne 0$ using rejection region
\[
R = \{ (x_1, \ldots, x_n) : |\bar{X}_n| > c \}
\]
Find and plot $\beta(\mu)$.

\begin{solution}
\[
\beta(\mu) = P_{\mu} (\|(\bar{X}_n) > c|) = P_{\mu} \left( \bar{X}_n > c \right) + P_{\mu} \left( \bar{X}_n < -c \right)
\]
\[
= 1 - \Phi(\sqrt{n}(c - \mu)) + \Phi(\sqrt{n}(-c - \mu))
\]
\end{solution}

\begin{lstlisting}[language=python]
import numpy as np
import matplotlib.pyplot as plt
from scipy.stats import norm
# parameters
n = 10
sigma = 1
c = 1
x = np.linspace(-5, 5, 400)
f = 1 - norm.cdf(np.sqrt(n) * (c-x) / sigma) + norm.cdf(-np.sqrt(n) * (c+x) / sigma)

\end{lstlisting}

\begin{figure}[H]
\centering
\includegraphics[width=0.4\textwidth]{images/reject.png}
\caption{}
\label{fig:}
\end{figure}
\end{eb}
% subsection Reject Rule (end)

\begin{eb}
Let $X \sim \text{Bin}(5, p)$. Test $H_0 : p \le \frac{1}{2}$ vs $H_1 : p > \frac{1}{2}$ with rejection regions:
\[
R_1 = \{x : x = 5\}, \quad R_2 = \{x : x \ge 3\}
\]
Plot and compare $\beta_1(p)$ and $\beta_2(p)$.

\begin{solution}
For a rejection region $R$, the power function is
\[
\beta(p) = P_p(X \in R).
\]

\noindent
For $R_1 = \{x = 5\}$,
\[
\beta_1(p) = P_p(X = 5) = {5 \choose 5} p^5 (1-p)^0 = p^5.
\]

\noindent
For $R_2 = \{x \ge 3\}$,
\[
\begin{aligned}
\beta_2(p) 
&= P_p(X \ge 3)
= \sum_{x=3}^{5} {5 \choose x} p^x (1-p)^{5-x} \\[6pt]
&= 10p^3(1-p)^2 + 5p^4(1-p) + p^5.
\end{aligned}
\]

\noindent
At $p = \tfrac{1}{2}$, the test sizes are
\[
\alpha_1 = \beta_1(0.5) = (0.5)^5 = 0.03125, 
\quad
\alpha_2 = \beta_2(0.5) = P_{0.5}(X \ge 3) = 0.5.
\]

Hence $R_2$ gives higher power but also much larger size.
\end{solution}


\begin{figure}[H]
\centering
\includegraphics[width=0.8\textwidth]{images/2R.png}
\caption{}
\label{fig:}
\end{figure}
\end{eb}


\subsection{Size and Level of a Test}

A test has \emph{level} $\alpha$ if its size $\le \alpha$, where
\[
\alpha = \sup_{\theta \in \Theta_0} \beta(\theta)
\]

That is, $\alpha$ is the largest probability of rejecting $H_0$ when $H_0$ is true (Type I error).

\[
\begin{array}{c|c|c}
 & \text{Fail to reject } H_0 & \text{Reject } H_0 \\ \hline
H_0 \text{ true} & \text{Correct} & \text{Type I error} \\
H_1 \text{ true} & \text{Type II error} & \text{Correct}
\end{array}
\]

\[
P_{H_0 \text{\ True}}(\text{Type I error}) = P_{H_0}(X\in R) \le \sup_{\theta \in \Theta_0} \beta(\theta) = \alpha
\]
\[
P_{H_1 \text{\ True}}(\text{Type II error}) = P_{H_1}(X \notin R) = 1 - P_{H_1}(X \in R) \le 1 - \inf_{\theta \in \Theta_1} \beta(\theta)
\]

