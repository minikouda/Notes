% TeX root = ../Main.tex

% First argument to \section is the title that will go in the table of contents. Second argument is the title that will be printed on the page.
\section[Lecture 15 (Oct 30) -- {\it Hypothesis Testing With Posterior}]{Lecture 15}

\subsection{Bayesian Hypothesis Testing}

In a Bayesian analysis, hypotheses, like parameters, can be described using
probability distributions.

The simplest case is when the hypotheses describe regions into which $\theta$
can fall, and these all have positive prior probability. If
$H_0 : \theta \in \Theta_0,$ then

\begin{itemize}
    \item \textbf{Prior probability:}
    \[
    P(H_0) = \int_{\Theta_0} f(\theta)\, d\theta
    \]

    \item \textbf{Posterior probability:}
    \[
    P(H_0 \mid x^n) = \int_{\Theta_0} f(\theta \mid x^n)\, d\theta
    \]
\end{itemize}

\bigskip

Suppose $H_0, \dots, H_{K-1}$ are $K$ hypotheses under consideration (typically
$K = 2$, but in theory we can have more). Suppose that under hypothesis $H_k$,
\[
\theta \sim f(\theta \mid H_k).
\]
Note that $\theta$ may mean different things under the various hypotheses.

Then,
\[
P(H_k \mid x^n)
=
\frac{f(x^n \mid H_k) P(H_k)}
{\sum_{k=1}^K f(x^n \mid H_k) P(H_k)}.
\]

Therefore, the posterior odds of $H_i$ relative to $H_j$ equals
\[
\frac{P(H_i \mid x^n)}{P(H_j \mid x^n)}
=
\frac{f(x^n \mid H_i)}{f(x^n \mid H_j)}
\times
\frac{P(H_i)}{P(H_j)}.
\]

The term $
\frac{f(x^n \mid H_i)}{f(x^n \mid H_j)} $
is called the \textbf{Bayes Factor} for comparing $H_i$ to $H_j$, and is denoted
by $\mathrm{BF}_{ij}$.

When $H_i$ and $H_j$ represent regions of the parameter space, it is often
easier to calculate the prior and posterior odds directly, and from these
compute the Bayes Factor.

If
\[
H_i : \theta = \theta_i
\quad \text{and} \quad
H_j : \theta = \theta_j,
\]
then the Bayes Factor is simply the ratio of likelihoods under the two values:
\[
\mathrm{BF}_{ij}
=
\frac{f(x^n \mid \theta_i)}{f(x^n \mid \theta_j)}.
\]

More generally,
\[
f(x^n \mid H_i)
=
\int_{\Theta} f(x^n \mid \theta, H_i)\, f(\theta \mid H_i)\, d\theta,
\]
which is called the \textbf{marginal likelihood}.

If $f(\theta \mid H_i)$ is conjugate, this integral can often be calculated in
closed form. Otherwise, we use sampling methods to approximate it. For example,
we could use Monte Carlo integration by sampling from $f(\theta \mid H_i)$.

\subsection{Hypothesis Testing using Posterior Odds}


\begin{eb}

Albert Pujols (St. Louis Cardinals) and Ichiro Suzuki (Seattle Mariners) had very similar batting averages over 2001--2010. Their career totals in that span were:
\[
\text{Pujols: } n=5146 \text{ at-bats, } x=1717 \text{ hits}\qquad
\text{Suzuki: } m=6099 \text{ at-bats, } y=2030 \text{ hits.}
\]
Let $X\mid p_1\sim \mathrm{Bin}(n,p_1)$ be Pujols' hits and $Y\mid p_2\sim \mathrm{Bin}(m,p_2)$ be Suzuki's hits. We wish to assess evidence for/against the hypothesis $p_{\text{Pujols}}=p_{\text{Suzuki}}$.

Under $H_0: p_1 = p_2$, $H_1:p_1\neq p_2$, assign independent priors $p_1\sim \mathrm{Unif}(0,1)$ and $p_2\sim \mathrm{Unif}(0,1)$. Compute the marginal likelihood
\[
f(x,y\mid H_1)=\int_0^1\!\int_0^1 f(x\mid p_1)\,f(y\mid p_2)\,dp_1\,dp_2.
\]

\end{eb}
\begin{solution}
With $X\mid p_1\sim\mathrm{Bin}(n,p_1)$ and $Y\mid p_2\sim\mathrm{Bin}(m,p_2)$,
\[
f(x\mid p_1,H_1)=\binom{n}{x}p_1^{x}(1-p_1)^{n-x},\qquad
f(y\mid p_2,H_1)=\binom{m}{y}p_2^{y}(1-p_2)^{m-y}.
\]
Using independence and the Uniform$(0,1)=\mathrm{Beta}(1,1)$ priors,
\[
\begin{aligned}
f(x,y\mid H_1)
&= \iint f(x_1\mid p_1, H_1) \, \underset{1}{f(p_1|H_1)}\,f(y\mid p_2,H_1) \underset{1}{f(p_2|H_1)}\,dp_1\,dp_2 \\ 
&=\binom{n}{x}\binom{m}{y}\!\left(\int_0^1 p_1^{x}(1-p_1)^{n-x}\,dp_1\right)
\left(\int_0^1 p_2^{y}(1-p_2)^{m-y}\,dp_2\right)\\
&=\binom{n}{x}\binom{m}{y}\,B(x+1,n-x+1)\,B(y+1,m-y+1),
\end{aligned}
\]
where $B(a,b)=\dfrac{\Gamma(a)\Gamma(b)}{\Gamma(a+b)}$ is the Beta function. Since
\[
\binom{n}{x}B(x+1,n-x+1)=\frac{n!}{x!(n-x)!}\cdot\frac{x!(n-x)!}{(n+1)!}=\frac{1}{n+1},
\]
and similarly for $m,y$, the marginal likelihood simplifies to
\[
\boxed{\,f(x,y\mid H_1)=\frac{1}{(n+1)(m+1)}\,}.
\]

\paragraph{Numerical value for these data.}
With $n=5146$ and $m=6099$,
\[
f(x,y\mid H_1)=\frac{1}{(5146+1)(6099+1)}=\frac{1}{31{,}396{,}700}\approx 3.19\times 10^{-8}.
\]

And for the null hypothesis $H_0:\;p_1=p_2=p$, with prior $p\sim \mathrm{Unif}(0,1)$,
\[\begin{aligned}
f(x,y\mid H_0)
&= \int_0^1 f(x,y\mid p,H_0)\,dp \\
&= \int_0^1 f(x\mid p,H_0)\,f(y\mid p,H_0)\,dp \\
&= \binom{n}{x}\binom{m}{y}\int_0^1 p^{x+y}(1-p)^{(n-x)+(m-y)}\,dp \\
&= \binom{n}{x}\binom{m}{y}B(x+y+1,n+m-(x+y)+1).
\end{aligned}\]

Let $p=P(H_1)$, so $P(H_0)=1-p$, and let $p^*=P(H_1\mid \text{Data})$.
By Bayes’ rule,
\[
p^* \;=\; \frac{f(\text{Data}\mid H_1)P(H_1)}
                {f(\text{Data}\mid H_1)P(H_1)+f(\text{Data}\mid H_0)P(H_0)}.
\]
Divide numerator and denominator by $f(\text{Data}\mid H_0)P(H_0)$:
\[
p^* \;=\; \frac{\displaystyle
\frac{f(\text{Data}\mid H_1)}{f(\text{Data}\mid H_0)}
\cdot \frac{p}{1-p}}
{\,1+\displaystyle
\frac{f(\text{Data}\mid H_1)}{f(\text{Data}\mid H_0)}
\cdot \frac{p}{1-p}}.
\]
Define the Bayes factor $BF_{10}=\dfrac{f(\text{Data}\mid H_1)}{f(\text{Data}\mid H_0)}$ to obtain
\[
\boxed{\;
p^* \;=\; \frac{\dfrac{p}{1-p}\,BF_{10}}
{1+\dfrac{p}{1-p}\,BF_{10}}
\;}
\]
which is equivalent to the odds form
\[
\frac{p^*}{1-p^*} \;=\; \frac{p}{1-p}\,BF_{10}.
\]

\end{solution}