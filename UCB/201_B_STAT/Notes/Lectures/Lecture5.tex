% TeX root = ../Main.tex

% First argument to \section is the title that will go in the table of contents. Second argument is the title that will be printed on the page.
\section[Lecture 5 (Sept. 11) -- {\it Sufficiency}]{Lecture 5}

\subsection{Sufficiency}

\subparagraph{Motivation} % (fold)
\label{subp:Motivation}
We hope to separate the information contained in the data into
the information relevant for making inference about $\theta$ and the information
irrelevant for these inferences. In other words, we would like to compress the
data to, e.g. $T(X)$, without loss of information. (Actually, it often turns
out that some part of the data carries no information about the unknown
distribution that produces the data)
% subparagraph Motivation (end)

\subparagraph{Benefits} % (fold)
\label{subp:Benefits}

1. increasing computational efficiency and decreasing storage requirements
2. involving irrelevant information may increase an estimator’s risk (see
Rao-Blackwell Theorem)
3. Improving the scientific interpretability of our data

% subparagraph Benefits (end)

\begin{eb}
Let $X_i \overset{i.i.d.}{\sim} Ber(\theta)$. Show that $T(X) = \sum\limits_{i=1}^{n} X_i$ is sufficient for $\theta$.

$$
P_0(X_1 = x_1, \dots, X_n = x_n | T(X) = t) = \dfrac{P_0(X_1 = x_1, \dots, X_n = x_n, T(X) = t)}{P_0(T(X) = t)}
$$
$$
= \dfrac{P_0(X_1 = x_1, \dots, X_n = x_n | \sum\limits_{i=1}^{n}X_i = t)}{P_0(\sum\limits_{i=1}^{n} X_i = t)} 
$$
$$
=
\left\{
\begin{array}{ll}
  0 & \text{\ when } t \ne \sum\limits_{i=1}^{n} x_i \\
  \dfrac{1}{\binom{n}{t}} & \text{ when } t = \sum\limits_{i=1}^{n} x_i \\
\end{array}
\right.
$$
\end{eb}

\begin{tb}[Neyman Factorization Theorem]
    Suppose the family
$\{P_\theta : \theta \in \Omega\}$ of distributions have joint mass functions or 
densities $\{p(x;\theta) : \theta \in \Omega\}$. Then a statistic $T$ is sufficient for 
$\theta$ if and only if there are functions $h$ and $g$ such that the density/mass 
function can be written

\[
p(x;\theta) = h(x)\, g(T(x), \theta).
\]
\end{tb}

\subparagraph{Proof} % (fold)
\label{subp:Proof}
$
\Rightarrow
$

If $T$ is sufficient for $\theta$, then
$$
P_\theta(X = x ) =  \underset{h(x)}{P_\theta(X=x | T(X) = T(x))}  \cdot P_\theta(T(X) = T(x))
$$
$$
= h(x) \cdot g(T(x), \theta)
$$
The first term is independent of $\theta$ according to the definition of Sufficient Statistics.

$
\Leftarrow
$
If $p(x;\theta) = h(x) g(T(x), \theta)$, then
$$
P_\theta(X = x | T(X) = t) = \dfrac{P_\theta(X = x, T(X) = t)}{P_\theta(T(X) = t)} = \dfrac{h(x) g(T(x), \theta)}{\sum\limits_{y: T(y) = t} h(y) g(T(y), \theta)}.
$$
Since $
P(X = x, T(X) = T(x)) = P(X = x)
$ and we need to run through all $y$ such that $T(y) = t$, the $g(T(y), \theta)$ term cancels out. So the conditional distribution does not depend on $\theta$.

$$
 = \dfrac{h(x)}{\sum\limits_{y: T(y) = t} h(y)}
$$
According to the definition of Sufficient Statistics, $T$ is sufficient for $\theta$.


% subparagraph Proof (end)


\begin{eb}
Let $X_i \sim U(0,\theta)$. Show that $T(X) = \max(X_1, \dots, X_n)$ is sufficient for $\theta$.

\begin{align*}
 p(x_1, \dots, x_n; \theta) & = \dfrac{1}{\theta^n} \cdot I(0 < x_1, \dots, x_n < \theta) = \dfrac{1}{\theta^n} I(0 < \max(x_1, \dots, x_n) < \theta) \\
 &   = \dfrac{1}{\theta^n} I(0 < Y_{(1)}) \cdot I(Y_{(n)} < \theta) \\
 & = I(Y_{(1)} > 0) \cdot \dfrac{1}{\theta^n} \cdot I(Y_{(n)} < \theta) \\
    & = h(Y) \cdot g(T(Y), \theta)
 \end{align*}
 $$
 T(Y) = Y_{(n)}
 $$
 

\end{eb}


\begin{tb}[The Rao-Blackwell Theorem]

Suppose $X$ is distributed according to $P_\theta(x) \in \{ P_\theta : \theta \in \Omega \}$ 
and a statistic $T(X)$ is sufficient for $\theta$. Given any estimator $\delta(X)$ of $\theta$, 
define
\[
\eta(T) = \mathbb{E}_\theta \big[ \delta(X) \mid T(X) \big].
\]
If the loss function $\mathcal{L}(\theta, \delta(X))$ is convex and the risk function
\[
R(\theta, \delta(X)) = \mathbb{E}\big[\mathcal{L}(\theta, \delta(X))\big] < \infty,
\]
then
\[
R(\theta, \eta) \leq R(\theta, \delta).
\]
If $\mathcal{L}$ is strictly convex, then the inequality is strict unless $\delta = \eta$.

\medskip
\noindent Note that the loss function reflects the degree of wrongness of an estimate. 
The commonly used quadratic loss function is defined as
\[
\mathcal{L}(\theta, \delta) = \big(\theta - \delta(X)\big)^2.
\]
\end{tb}
\begin{proof}

$\delta(x)$: an estimator of $\theta$.

$\eta(x) := \mathbb{E}_\theta \big[ \delta(X) \mid T(X) \big] = \eta (T(X))$ a function of $T(X)$.

$$
E_{\theta,x}[\eta(x) | T(x)] = E_{\theta,x|T(x)}[ \eta(x)] = \int \eta(x) f(x|T(x)) dx \text{\ no theta}
$$

$\mathbb{E}_\theta(\eta(x)) = \mathbb{E}_\theta \big[ \mathbb{E}_\theta \big[ \delta(X) \mid T(X) \big] \big] = \mathbb{E}_\theta \big[ \delta(X) \big]$

$\mathcal{L}(\theta, \eta) $ loss function

$R(\theta,\delta) = \mathbb{E}_\theta(\mathcal{L}(\theta, \delta(X)))$

\begin{lb}[Jensen Inequality]
Let $(\Omega, \mathcal{F}, \mathbb{P})$ be a probability space, and let 
$X : \Omega \to \mathbb{R}$ be an integrable random variable, i.e.\ $E[|X|] < \infty$.

Let $\varphi : \mathbb{R} \to \mathbb{R}$ be a convex function such that $\varphi(X)$ is integrable. Then
\[
\varphi\!\big(E[X]\big) \;\leq\; E[\varphi(X)].
\]

Moreover, if $\varphi$ is strictly convex, then equality holds if and only if $X$ is almost surely constant.

\begin{proof}
Let $a = \mathbb{E}[X]$. By the definition of convexity, for any $x$,
\[
\phi(x) \geq \phi(a) + \phi'(a)(x - a).
\]
Taking expectation on both sides gives
\[
\mathbb{E}[\phi(X)] \geq \phi(a) + \phi'(a)(\mathbb{E}[X] - a) = \phi(a).
\]
\end{proof}
\end{lb}

$$
R(\theta, \eta) = \mathbb{E}_\theta(\mathcal{L}(\theta, \eta(X))) = \mathbb{E}_\theta(\mathcal{L}(\theta, \eta(T(X))))
$$
$$
= \E_{\theta,x} [ L(\theta,E_{\theta,x}[\delta(X) | T(X)]) ] = \E_{\theta,x} [ \mathcal{L}(\theta,E_{\theta,x|T(X)}[\delta(X)]) ] 
$$
$$
\le \E_{\theta,x} [ E_{\theta,x|T(X)}[\mathcal{L}(\theta,\delta(X))] ] \qquad \text{Jensen Inequality}
$$
$$
= \E_{\theta,x} [\mathcal{L}(\theta,\delta(X))] = R(\theta,\delta(x)) \qquad \text{Law of iterated expectation}
$$



\end{proof}



