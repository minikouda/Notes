%%%%%%%%%%%%%%%%%%%%%%%%%%%%%%%%%%%%%%%%%
% fphw Assignment
% LaTeX Template
% Version 1.0 (27/04/2019)
%
% This template originates from:
% https://www.LaTeXTemplates.com
%
% Authors:
% Class by Felipe Portales-Oliva (f.portales.oliva@gmail.com) with template 
% content and modifications by Vel (vel@LaTeXTemplates.com)
%
% Template (this file) License:
% CC BY-NC-SA 3.0 (http://creativecommons.org/licenses/by-nc-sa/3.0/)
%
%%%%%%%%%%%%%%%%%%%%%%%%%%%%%%%%%%%%%%%%%

%----------------------------------------------------------------------------------------
%	PACKAGES AND OTHER DOCUMENT CONFIGURATIONS
%----------------------------------------------------------------------------------------

\documentclass[
  12pt,
  %fleqn, % Default font size, values between 10pt-12pt are allowed
  %letterpaper, % Uncomment for US letter paper size
  %spanish, % Uncomment for Spanish
]{fphw}

% Template-specific packages
\usepackage[utf8]{inputenc} % Required for inputting international characters
\usepackage[T1]{fontenc} % Output font encoding for international characters
\usepackage{mathpazo} % Use the Palatino font

\usepackage{graphicx} % Required for including images

\usepackage{booktabs} % Required for better horizontal rules in tables

\usepackage{listings} % Required for insertion of code

\usepackage{enumerate} % To modify the enumerate environment

\usepackage{enumitem}

\usepackage{amsmath}

\usepackage{mathtools,amsthm}

\usepackage{float}

\usepackage{listings}

\usepackage{hyperref}

\lstdefinelanguage{R}{
  keywords={if, else, repeat, while, function, for, in, next, break},
  otherkeywords={TRUE, FALSE, NULL, NA, Inf, NaN},
  sensitive=true,
  morecomment=[l]\#,
  morestring=[b]",
}

\newcommand{\E}{\mathbb{E}}
\newcommand{\Var}{\operatorname{Var}}
\newcommand{\R}{\mathbb{R}}
\newcommand{\norm}[1]{\left\lVert #1 \right\rVert}


%----------------------------------------------------------------------------------------
%	ASSIGNMENT INFORMATION
%----------------------------------------------------------------------------------------

\title{Homework \#1} % Assignment title

\author{Shizhe Zhang} % Student name

\date{\today} % Due date

\institute{University of California, Berkeley \\ Department of Statistics} % Institute or school name

\class{EECS 227AT} % Course or class name

\professor{Gireeja Ranade} % Professor or teacher in charge of the assignment

%----------------------------------------------------------------------------------------

\begin{document}

\maketitle % Output the assignment title, created automatically using the information in the custom commands above

%----------------------------------------------------------------------------------------
%	ASSIGNMENT CONTENT
%----------------------------------------------------------------------------------------

\section*{Problem 1. Course Setup}
\begin{problem}

Please complete the following steps to get access to all course resources.
\begin{enumerate}
    \item[(a)] Visit the course website at \texttt{http://eecs127.github.io/} and familiarize yourself with the syllabus.
    \item[(b)] Verify that you can access the class Ed site at \texttt{https://edstem.org/us/courses/93760/discussion}.
    \item[(c)] Verify that you can access the class Gradescope site.
\end{enumerate}
\end{problem}

\subsection*{Answer}


%----------------------------------------------------------------------------------------

\section*{Problem 2. Prerequisites}
\begin{problem}
The prerequisites for this course are:
\begin{itemize}
    \item MATH 54 (Linear Algebra \& Differential Equations),
    \item CS 70 (Discrete Mathematics \& Probability Theory),
    \item MATH 53 (Multivariable Calculus).
\end{itemize}

Please fill out the Google form provided on the homework PDF.  
For your written response, write the \emph{secret word} revealed at the end of the form.
\end{problem}

\subsection*{Answer}
PinkpantheR

%----------------------------------------------------------------------------------------

\newpage
\section*{Problem 3. Orthogonality}
\begin{problem}

Let $\vec{x}, \vec{y} \in \mathbb{R}^n$ be two linearly independent unit-norm vectors, i.e.,
\[
\|\vec{x}\|_2 = \|\vec{y}\|_2 = 1.
\]

\begin{enumerate}
    \item[(a)] Show that the vectors $\vec{u} = \vec{x} - \vec{y}$ and $\vec{v} = \vec{x} + \vec{y}$ are orthogonal.
    \item[(b)] Find an orthonormal basis for $\text{span}(\vec{x}, \vec{y})$.
\end{enumerate}
\end{problem}

\subsection*{Answer}
\paragraph{(a)} % (fold)
To show that $\vec{u}$ and $\vec{v}$ are orthogonal, we need to show that their dot product is zero:
\[
\vec{u} \cdot \vec{v} = (\vec{x} - \vec{y}) \cdot (\vec{x} + \vec{y}).
\]
Expanding the dot product, we have:
\[
\vec{u} \cdot \vec{v} = \vec{x} \cdot \vec{x} + \vec{x} \cdot \vec{y} - \vec{y} \cdot \vec{x} - \vec{y} \cdot \vec{y}.
\]

Since the dot product is commutative, $\vec{x} \cdot \vec{y} = \vec{y} \cdot \vec{x}$. Also, given that $\|\vec{x}\|_2 = \|\vec{y}\|_2 = 1$, we have $\vec{x} \cdot \vec{x} = 1$ and $\vec{y} \cdot \vec{y} = 1$. Therefore,

\[
\vec{u} \cdot \vec{v} = 1 + \vec{x} \cdot \vec{y} - \vec{x} \cdot \vec{y} - 1 = 0. \Rightarrow \vec{u} \perp \vec{v}
\]


% paragraph (a) (end)

\paragraph{(b)} % (fold)

We use the Gram-Schmidt process.
Starting with the vectors $\vec{x}$ and $\vec{y}$.
First, we take $\vec{u}_1 = \vec{x}$. Next, we project $\vec{y}$ onto $\vec{u}_1$:

\[\vec{p}_{\vec{y}}  = \frac{\vec{y} \cdot \vec{u}_1}{\|\vec{u}_1\|_2^2} \vec{u}_1 = (\vec{y} \cdot \vec{x}) \vec{x}.\]
Now, we subtract this projection from $\vec{y}$ to get $\vec{u}_2$:
\[\vec{u}_2 = \vec{y} - \vec{p}_{\vec{y}} = \vec{y} - (\vec{y} \cdot \vec{x}) \vec{x}.\]
Next, we normalize $\vec{u}_1$ and $\vec{u}_2$ to get the orthonormal basis vectors:
\[\vec{e}_1 = \frac{\vec{u}_1}{\|\vec{  u}_1\|_2} = \vec{x},\]
\[\vec{e}_2 = \frac{\vec{u}_2}{\|\vec{u}_2\|_2} = \frac{\vec{y} - (\vec{y} \cdot \vec{x}) \vec{x}}{\|\vec{y} - (\vec{y} \cdot \vec{x}) \vec{x}\|_2}.\]
Thus, the orthonormal basis for $\text{span}(\vec{x}, \vec{y})$ is given by the vectors $\vec{e}_1$ and $\vec{e}_2$:
\[\left\{ \vec{x}, \frac{\vec{y} - (\vec{y} \cdot \vec{x}) \vec{x}}{\|\vec{y} - (\vec{y} \cdot \vec{x}) \vec{x}\|_2} \right\}.\]

% paragraph (b) (end)

%----------------------------------------------------------------------------------------

\section*{Problem 4. Least Squares}
\begin{problem}
The Michaelis--Menten model for enzyme kinetics relates the rate $y$ of an enzymatic reaction
to the concentration $x$ of a substrate, as follows:
\[
y = \frac{\beta_1 x}{\beta_2 + x}, \tag{1}
\]
for constants $\beta_1, \beta_2 > 0$.

\begin{enumerate}
    \item[(a)] Show that the model can be expressed as a linear relation between the values
    $1/y = y^{-1}$ and $1/x = x^{-1}$.
    Specifically, give an equation of the form
    \[
    y^{-1} = w_1 + w_2 x^{-1},
    \]
    specifying the values of $w_1$ and $w_2$ in terms of $\beta_1$ and $\beta_2$.

    \item[(b)] In general, reaction parameters $\beta_1$ and $\beta_2$
    (and, thus, $w_1$ and $w_2$) are not known a priori and must be fit from data,
    for example using least squares.
    Suppose you collect $m$ measurements $(x_i, y_i)$, $i = 1, \ldots, m$,
    over the course of a reaction.

    Formulate the least squares problem
    \[
    \vec{w}^\star = \arg\min_{\vec{w}} \|X\vec{w} - \vec{y}\|_2^2, \tag{2}
    \]
    where
    \[
    \vec{w}^\star =
    \begin{bmatrix}
    w_1^\star \\
    w_2^\star
    \end{bmatrix},
    \]
    and you must specify $X \in \mathbb{R}^{m \times 2}$ and $\vec{y} \in \mathbb{R}^m$.
    Specifically, your solution should include explicit expressions for $X$ and $\vec{y}$
    as a function of $(x_i, y_i)$ values and a final expression for $\vec{w}^\star$
    in terms of $X$ and $\vec{y}$, which should contain only matrix multiplications,
    transposes, and inverses.

    Assume without loss of generality that $x_1 \neq x_2$.

    \item[(c)] Assume that we have used the above procedure to calculate values for
    $w_1^\star$ and $w_2^\star$, and we now want to estimate
    \[
    \hat{\boldsymbol{\beta}} =
    \begin{bmatrix}
    \hat{\beta}_1 \\
    \hat{\beta}_2
    \end{bmatrix}.
    \]
    Write an expression for $\hat{\boldsymbol{\beta}}$ in terms of $w_1^\star$ and $w_2^\star$.
\end{enumerate}

\textbf{NOTE:} This problem was taken (with some edits) from the textbook
\emph{Optimization Models} by Calafiore and El Ghaoui.
\end{problem}

\subsection*{Answer}

\paragraph{(a)} % (fold)
Taking the reciprocal of both sides of equation (1), we have:
\[y^{-1} = \frac{\beta_2 + x}{\beta_1 x} = \frac{\beta_2}{\beta_1 x} + \frac{x}{\beta_1 x} = \frac{\beta_2}{\beta_1} x^{-1} + \frac{1}{\beta_1}.\]
Thus, we can express the model as a linear relation between $y^{-1}$ and $x^{-1}$:
\[y^{-1} = \frac{1}{\beta_1} + \frac{\beta_2}{\beta_1} x^{-1}.\]
Therefore, we have:
\[w_1 = \frac{1}{\beta_1}, \quad w_2 = \frac{\beta_2}{\beta_1}.\]
% paragraph (a) (end)

\paragraph{(b)} % (fold)
To formulate the least squares problem, we first need to express the measurements in terms of $y^{-1}$ and $x^{-1}$. For each measurement $(x_i, y_i)$, we have:
\[y_i^{-1} = w_1 + w_2 x_i^{-1}.\]
We can rewrite this in matrix form as:
\[\begin{bmatrix}
y_1^{-1} \\
y_2^{-1} \\
\vdots \\
y_m^{-1}
\end{bmatrix} = \begin{bmatrix}
1 & x_1^{-1} \\
1 & x_2^{-1} \\
\vdots & \vdots \\
1 & x_m^{-1}
\end{bmatrix} \begin{bmatrix}
w_1 \\
w_2
\end{bmatrix}.\]
Thus, we can define:
\[X = \begin{bmatrix}
1 & x_1^{-1} \\
1 & x_2^{-1} \\ 
\vdots & \vdots \\
1 & x_m^{-1}
\end{bmatrix} \in \mathbb{R}^{m \times 2}, \quad \vec{y} = \begin{bmatrix}
y_1^{-1} \\
y_2^{-1} \\
\vdots \\
y_m^{-1}
\end{bmatrix} \in \mathbb{R}^m.\]
The least squares problem can then be expressed as:
\[\vec{w}^\star = \arg\min_{\vec{w}} \|X\vec{w} - \vec{y}\|_2^2.\]
The solution to this least squares problem is given by the normal equation:
\[\vec{w}^\star = (X^\top X)^{-1} X^\top \vec{y}.\]

% paragraph (b) (end)

\paragraph{(c)}
From part (a),
\[
w_1=\frac{1}{\beta_1}
\quad\Rightarrow\quad
\beta_1=\frac{1}{w_1},
\qquad
w_2=\frac{\beta_2}{\beta_1}
\quad\Rightarrow\quad
\beta_2=\beta_1 w_2=\frac{w_2}{w_1}.
\]
\[
\hat{\boldsymbol{\beta}}
=
\begin{bmatrix}
\hat{\beta}_1\\
\hat{\beta}_2
\end{bmatrix}
=
\begin{bmatrix}
\dfrac{1}{w_1^\star}\\[6pt]
\dfrac{w_2^\star}{w_1^\star}
\end{bmatrix}.
\]

\newpage
\section*{Problem 5. Subspaces and Dimensions}
\begin{problem}
Consider the set $S$ of points $(x_1, x_2, x_3) \in \mathbb{R}^3$ such that
\[
x_1 + 2x_2 + 3x_3 = 0, \quad
3x_1 + 2x_2 + x_3 = 0. \tag{3}
\]

\begin{enumerate}
    \item[(a)] Find a $2 \times 3$ matrix $A$ for which $S$ is exactly the null space of $A$.
    \item[(b)] Determine the dimension of $S$ and find a basis for it.
\end{enumerate}
\end{problem}

\subsection*{Answer}

\paragraph{(a)} % (fold)
We can express the given equations in matrix form as follows:
\[\begin{bmatrix}
1 & 2 & 3 \\
3 & 2 & 1
\end{bmatrix}
\begin{bmatrix}
x_1 \\
x_2 \\
x_3
\end{bmatrix} = \mathbf{0}.\]
Thus, the matrix $A$ is given by:
\[A = \begin{bmatrix}
1 & 2 & 3 \\
3 & 2 & 1
\end{bmatrix}.\]
% paragraph (a) (end)

\paragraph{(b)} % (fold)
To determine the dimension of $S$ and find a basis for it, we first need to find the null space of the matrix $A$.
We can perform row reduction on the matrix $A$:
\[\begin{bmatrix}
1 & 2 & 3 \\
3 & 2 & 1
\end{bmatrix} \xrightarrow{R_2 \leftarrow R_2 - 3R_1} \begin{bmatrix}
1 & 2 & 3 \\
0 & -4 & -8
\end{bmatrix} \xrightarrow{R_2 \leftarrow -\frac{1}{4} R_2} \begin{bmatrix}
1 & 2 & 3 \\
0 & 1 & 2
\end{bmatrix}.\]
Now, we can express the system of equations represented by the row-reduced matrix:
\[x_1 + 2x_2 + 3x_3 = 0,\]
\[x_2 + 2x_3 = 0.\]
From the second equation, we have:
\[x_2 = -2x_3.\]
Substituting this into the first equation, we get:
\[x_1 + 2(-2x_3) + 3x_3 = 0 \implies x_1 - 4x_3 + 3x_3 = 0 \implies x_1 - x_3 = 0 \implies x_1 = x_3.\]
Letting $x_3 = t$, we can express the solution as:
\[\begin{bmatrix}
x_1 \\
x_2 \\
x_3
\end{bmatrix} = \begin{bmatrix}
t \\
-2t \\
t
\end{bmatrix} = t \begin{bmatrix}
1 \\
-2 \\
1
\end{bmatrix}.\]
Thus, the null space of $A$ is spanned by the vector $\begin{bmatrix}
1 \\ -2 \\ 1
\end{bmatrix}$, and the dimension of $S$ is 1.
Therefore, a basis for $S$ is given by:
\[\left\{ \begin{bmatrix}
1 \\ -2 \\ 1
\end{bmatrix} \right\}.\]
% paragraph (b) (end)


\section*{Problem 6. Vector Spaces and Rank}
\begin{problem}
The rank of a $m \times n$ matrix $A$, $\mathrm{rank}(A)$, is the dimension of its range,
also called span, and denoted
\[
\mathcal{R}(A) := \{A\vec{x} : \vec{x} \in \mathbb{R}^n\}.
\]

\begin{enumerate}
    \item[(a)] Assume that $A \in \mathbb{R}^{m \times n}$ takes the form
    \[
    A = \vec{u}\vec{v}^\top,
    \]
    with $\vec{u} \in \mathbb{R}^m$, $\vec{v} \in \mathbb{R}^n$, and
    $\vec{u}, \vec{v} \neq \vec{0}$.
    (Note that a matrix of this form is known as a dyad.)
    Find the rank of $A$.

    \textbf{HINT:} Consider the quantity $A\vec{x}$ for arbitrary $\vec{x}$.

    \item[(b)] Show that for arbitrary $A, B \in \mathbb{R}^{m \times n}$,
    \[
    \mathrm{rank}(A + B) \le \mathrm{rank}(A) + \mathrm{rank}(B). \tag{4}
    \]

    \textbf{HINT:} First, show that $\mathcal{R}(A + B) \subseteq \mathcal{R}(A) + \mathcal{R}(B)$.
    Remember that for any matrix $A$, $\mathcal{R}(A)$ is a subspace, and for any two subspaces
    $S_1$ and $S_2$, $\dim(S_1 + S_2) \le \dim(S_1) + \dim(S_2)$.
    The sum of vector spaces is defined as
    \[
    S_1 + S_2 := \{\vec{s}_1 + \vec{s}_2 \mid \vec{s}_1 \in S_1, \vec{s}_2 \in S_2\}.
    \]

    \item[(c)] Consider an $m \times n$ matrix $A$ that takes the form
    \[
    A = UV^\top,
    \]
    with $U \in \mathbb{R}^{m \times k}$ and $V \in \mathbb{R}^{n \times k}$.
    Show that the rank of $A$ is less than or equal to $k$.

    \textbf{HINT:} Use parts (a) and (b), and remember that this decomposition can also be written as
    the dyadic expansion
    \[
    A = UV^\top =
    \begin{bmatrix}
    \vec{u}_1 & \cdots & \vec{u}_k
    \end{bmatrix}
    \begin{bmatrix}
    \vec{v}_1^\top \\
    \vdots \\
    \vec{v}_k^\top
    \end{bmatrix}
    =
    \sum_{i=1}^k \vec{u}_i \vec{v}_i^\top. \tag{5}
    \]
\end{enumerate}
\end{problem}

\subsection*{Answer}

\paragraph{(a)} % (fold)
\[A\vec{x} = \vec{u}\vec{v}^\top \vec{x} = \vec{u}(\vec{v}^\top \vec{x}). \]
The term $\vec{v}^\top \vec{x}$ is a scalar, so we can denote it as $c = \vec{v}^\top \vec{x}$. Therefore,
\[A\vec{x} = c\vec{u}.\]
This shows that the image of any vector $\vec{x}$ under the transformation defined by $A$ is a scalar multiple of the vector $\vec{u}$. Hence, the range of $A$ is spanned by the single vector $\vec{u}$.
Since $\vec{u} \neq \vec{0}$, the range of $A$ is one-dimensional. Therefore, the rank of $A$ is:
\[\mathrm{rank}(A) = 1.\]
% paragraph (a) (end)

\paragraph{(b)} % (fold)

To show that $\mathrm{rank}(A + B) \le \mathrm{rank}(A) + \mathrm{rank}(B)$, we first need to show that $\mathcal{R}(A + B) \subseteq \mathcal{R}(A) + \mathcal{R}(B)$.
Let $\vec{y} \in \mathcal{R}(A + B)$. Then, by definition of the range, there exists a vector $\vec{x} \in \mathbb{R}^n$ such that:
\[\vec{y} = (A + B)\vec{x} = A\vec{x} + B\vec{x}.\]

Since $A\vec{x} \in \mathcal{R}(A)$ and $B\vec{x} \in \mathcal{R}(B)$, thus:
$\mathcal{R}(A + B) \subseteq \mathcal{R}(A) + \mathcal{R}(B)$.

Now, using the property of dimensions of subspaces, we have:
\[\dim(\mathcal{R}(A + B)) \le \dim(\mathcal{R}(A) + \mathcal{R}(B)) \le \dim(\mathcal{R}(A)) + \dim(\mathcal{R}(B)).\]

Therefore,
\[\mathrm{rank}(A + B) \le \mathrm{rank}(A) + \mathrm{rank}(B).\]
% paragraph (b) (end)

\paragraph{(c)} % (fold)
Using the dyadic expansion of $A$ from equation (5), we can express $A$ as a sum of $k$ dyads:
\[A = \sum_{i=1}^k \vec{u}_i \vec{v}_i^\top.\]
From part (a), we know that each dyad $\vec{u}_i \vec{v}_i^\top$ has rank 1. Therefore, we can apply the result from part (b) to find the rank of $A$:
\[\mathrm{rank}(A) = \mathrm{rank}\left(\sum_{i=1}^k \vec{u}_i \vec{v}_i^\top\right) \le \sum_{i=1}^k \mathrm{rank}(\vec{u}_i \vec{v}_i^\top) = \sum_{i=1}^k 1 = k.\]
Thus, we have shown that the rank of $A$ is less than or equal to $k$:
\[\mathrm{rank}(A) \le k.\]
% paragraph (c) (end) 


\section*{Problem 7. Homework Process}
\begin{problem}
With whom did you work on this homework?
List the names and SIDs of your group members.

\textbf{NOTE:} If you didn’t work with anyone, you can put ``none'' as your answer.
\end{problem}

\subsection*{Answer}
none


\end{document}